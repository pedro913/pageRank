\documentclass[12pt]{article}

\usepackage[utf8]{inputenc}
\usepackage{amsmath,amsfonts,amsthm,amssymb,bezier}
\usepackage[brazil]{babel}
\usepackage[onehalfspacing]{setspace}
\usepackage{indentfirst}

\setlength{\textwidth}{6.5in} \setlength{\textheight}{9.5in}
\setlength{\topmargin}{-0.65in} \setlength{\evensidemargin}{-1mm}
\setlength{\oddsidemargin}{-1mm}
\setlength{\parskip}{0.3\baselineskip}
\setlength{\fboxsep}{0.3cm}
\linespread{1.05}
\setlength\parindent{.25in}
	
	\title{\textbf{Relatório do Projeto de MAP2110}}
	\author{Pedro Henrique Alves de Queiroz 11811180}
	\date{\today}

\begin{document}
    
\maketitle

	
\section{Introdução}
    A Google é uma empresa de década de 90 que conseguiu notorio destaque ... Neste relatório explicaremos os métodos utilizados do, bem como as ideias para construção do programa, complexidade, visão geral dos métodos, além da criação de testes e análise deles. 

\vspace{5em}

\section{Métodos}
	Nessa seção faremos um levantamento dos dois métodos utilizados no projeto, comentando as adaptações feitas, possíveis otimizações se tiver um tipo de grafo já determinado na entrada, além de uma comparação dos métodos utilizados. 
	
	Inicialmente temos que a 
	 \subsection{Método 1 - Escalonamento}
    	\subsubsection{Introdução}
    		Esse método consciste em dada uma matriz de ligação $M$ $n\times n$ transformamos $M$ em:
    	
    		\vspace{1em}
    		\centerline{$M = (1-\alpha)M + \alpha S_n$,} 
    		\vspace{0.3em}
    		
    	 	onde $\alpha$ é um valor entre 0 e 1 e $S_n$ é uma matriz com todos um valores iguais a $1/n$, afim de evitar possivéis colunas de zero ("quando temos uma página que não aponta para ninguém"). onde $\alpha$ é um valor entre 0 e 1 e $S_n$ é uma matriz com todos um valores iguais a $1/n$, afim de evitar possivéis colunas de zero ("quando temos uma página que não aponta para ninguém").
    	\subsubsection{Possíveis adaptações/otimizações}
    	
    \subsection{Método 2}

\vspace{5em}
   
\section{Tarefas}

	
\end{document}
